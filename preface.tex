\section*{まえがき}

これは Michael Spivak ``Calculus on Manifolds: A Modern Approach to Classical Theorems of Advanced Calculus'' (CRC Press, 1971) および齋藤正彦による訳書「スピヴァック 多変数の解析学―古典理論への現代的アプローチ」(東京図書, 2007)に沿って,多変数解析学の勉強会をした(する)ときのノートである.大筋としては Spivak に沿った進行をしているが,詳細は異なる.記法や用語といった細かな点については勿論のこと,証明も私個人の嗜好を優先して書いたものが多い.特に積分は測度論に基づいており,加藤賢悟「測度論的確率論 講義ノート 2016年版」を参考にしている.勉強会の進行に合わせて内容は随時加筆される.前提知識としては集合と写像の記号に関する初歩的な取り扱い,1変数実関数の微分積分,および行列と行列式の計算ができることを仮定しているが,実際のところは線型写像とその行列表示の概念までは理解していたほうが読みやすいと思う.

より具体的には,多変数関数の微積分の定義からはじめて,つぎの形の Stokes の定理およびその周辺の話題までをカバーすることを最終的な目標にしている.
\begin{thm*}
$M$を向きづけられた$k$次元境界つきコンパクト微分可能多様体,$\omega$を$M$上定義された微分$k-1$-形式とするとき,
\begin{equation}
\int_M d\omega = \int_{\partial M} \omega
\end{equation}がなりたつ.ただし,$\partial M$には$M$から誘導される向きを入れるものとする.特に,$M$として$\Real$の閉区間$[a,b]$,$\omega$として$[a,b]$上の0-形式(すなわち$C^\infty$級関数)$F$を取り,$F$の導関数を$f$と書くことにすれば,
\begin{equation}
\int_a^b f(x)\ dx = F(b) - F(a)
\end{equation}がなりたつ.
\end{thm*}
定理の後半のステートメントは「微積分学の基本定理」と呼ばれているものにほかならない.いわゆる現代幾何学の入り口が,実際には初等的な解析で学ぶ概念と地続きになっていることをきちんと理解したいというのがこのノートを書くに至った動機のひとつである.

本文の議論とは関係ないかもしれないが,コメントしておくに値すると思った事柄を「余談」として設けてある.また,本文中のそこかしこに「問」が載っているが,読者に対して解くことを要求しているというよりは,私が証明や説明を省いた事柄がすべて「問」扱いされている,というほうが正確であろう.なお,省いた理由はだいたいが「私が書くのをめんどくさがった」ないしは「私が答えを知らない,知りたい」のどちらかである.それゆえ,「〜を示せ」という書き方はしていない.そのような立ち位置の「問」ではあるが,解いてもらっても構わないし,いい感じに解けたと思ったらぜひ教えてほしい.また,「問」に Spivak の章末問題を引用することはできるだけ避けてあり,私の手元で解けた問題についてはその解答を定理・命題などの形で本文に書くようにした\footnote{のだが,どれが Spivak の章末問題を解いた結果の命題だったか,ちゃんと記録するのを怠ってしまい,書いた私自身もわからなくなっている.気が向いたら対応付けを調べて明記します.}.

時に,このノートの趣旨から大幅に逸脱した「問」が掲載されていることがある.そのような「問」は星印 (*) を付した.逸脱度合いがひどいものは二つ星 (**) にした.これらは私が気になっている(が,ちゃんと勉強したことのない)事柄である.これらの「問」はつっこみが入らない限り勉強会で扱うことは稀であり,読み飛ばしても本文を読む上では問題ないようにしてある(はずである).特に,私が解答のあらすじを知っているとは限らないこと,本ノートにおける前提知識などを全く踏まえていないことを注意してほしい.これらの「問」に解答をつけるつもりもとりあえずはないが,どこかで勉強したいと思っていることは事実である.